\section{Polynomial Curve Fitting}
\subsection{\textcolor{teal}{Python Code}}

\begin{minted}[frame=lines, linenos, fontsize=\large]
{python}
import pandas as pd
import numpy as np
from mpl_toolkits.mplot3d import Axes3D
import matplotlib.pyplot as plt
import random
from matplotlib import cm

def process_dataset_1(degree, lamb, batch):
    
    train = int(0.7*batch)   #70% of data
    valid = int(0.2*batch)   #20 % of data
    test = int(batch-(train+valid))  #10 % of data   
    
    #generating polynomial combinations
    def polybasisfun(vec,deg):
        def degcomb(vec,deg):
            if deg == 0:
                return [1]
            if deg == 1:
                return vec
            if len(vec) == 1:
                return [vec[0]**deg]
            u = []
            for i in range(deg+1):
                u+=([(vec[0]**(i))*x for x in (degcomb(vec[1:],deg-i))])
            return u
        u = np.array([1])
        for i in range(1,deg+1):
            u=np.append(u,degcomb(vec,i))
        return u
    
    # Read data from csv file
    f0 = pd.read_csv('datasets/function1.csv')
    odata = f0.to_numpy()
    np.random.shuffle(odata)
    
    # split into train,validate,test
    data = odata[:train,1:2]
    validata = odata[-(valid):,1:2]
    testdata = odata[-(valid+test):-(valid),1:2]
    yd = odata[:train,2]
    yv = odata[-(valid):,2]
    yt = odata[-(valid+test):-(valid),2]
    
    # formulate X matrix
    X = np.apply_along_axis(polybasisfun,1,data,degree)
    
    # calculate weights
    w = ((np.linalg.inv((lamb*np.eye(np.shape(X)[1]))+((np.transpose(X))@X)))@((np.transpose(X))@yd))
    
    
    # Get predictions
    def output(v,deg,w):
        return (w@(np.transpose(polybasisfun(v,deg))))
    yp = np.apply_along_axis(output,1,data,deg=degree,w=w)
    yvp = np.apply_along_axis(output,1,validata,deg=degree,w=w)
    ytp = np.apply_along_axis(output,1,testdata,deg=degree,w=w)
    
    
    print('ERMS_training = %f'%(np.linalg.norm((yp-yd),2)/np.sqrt(train)))
    print('ERMS_test = %f'%(np.linalg.norm((ytp-yt),2)/np.sqrt(test)))
    print('ERMS_valid = %f'%(np.linalg.norm((yvp-yv),2)/np.sqrt(valid)))
    
    
    x = np.linspace(min(data),max(data),50)
    x = np.reshape(x,(x.shape[0],1))
    y = np.apply_along_axis(lambda x:output(x,degree,w),1,x)
    
    
    fig0 = plt.figure(figsize=(9,9))
    plt.plot(x,y,color='red')
    plt.scatter(data,yd,color='blue')
    plt.title("degree = {} | lambda = {}".format(degree,lamb),fontsize = 20)
    plt.xlabel('X-axis(x)')
    plt.ylabel('Y-axis(y)')
    
    
    return np.linalg.norm((yp-yd),2)/np.sqrt(train), np.linalg.norm((ytp-yt),2)/np.sqrt(test), np.linalg.norm((yvp-yv),2)/np.sqrt(valid)
    
################################# Main Code ##################################

# # Call function to experiment dataset 1
Degree = [2,3,6,9]
lamb = [0,10e-2,10e-1,10,10e2]
batch = [10,200]

Erms = []
# Experiment with Degree
for l in lamb:
    erms = process_dataset_1(25, l, 350)
    Erms.append(erms)
    

\end{minted}
