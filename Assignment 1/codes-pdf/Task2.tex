\section{Linear Model for Regression using Polynomial Basis Functions}
\subsection{\textcolor{teal}{Python Code}}

\begin{minted}[frame=lines, linenos, fontsize=\large]
{python}

import pandas as pd
import numpy as np
from mpl_toolkits.mplot3d import Axes3D
import matplotlib.pyplot as plt
import random
from matplotlib import cm
#polynomial basis function regression
degree = 6
lamb = 0
batch = 500 
train = int(0.7*batch)   #70% of data
test = int(0.2*batch)   #20 % of data
valid = batch-(train+test)  #10 % of data   


#generating polynomial combinations
def polybasisfun(vec,deg):
    def degcomb(vec,deg):
        if deg == 0:
            return [1]
        if deg == 1:
            return vec
        if len(vec) == 1:
            return [vec[0]**deg]
        u = []
        for i in range(deg+1):
            u+=([(vec[0]**(i))*x for x in (degcomb(vec[1:],deg-i))])
        return u
    u = np.array([1])
    for i in range(1,deg+1):
        u=np.append(u,degcomb(vec,i))
    return u
f0 = pd.read_csv('function1_2d.csv')
odata = f0.to_numpy()
np.random.shuffle(odata)
data = odata[:train,1:3]
validata = odata[-(valid):,1:3]
testdata = odata[-(valid+test):-(valid),1:3]
yv = odata[-(valid):,3]
yt = odata[-(valid+test):-(valid),3]
X = np.apply_along_axis(polybasisfun,1,data,degree)
yd = odata[:train,3]
w = ((np.linalg.inv((lamb*np.eye(np.shape(X)[1]))+((np.transpose(X))@X)))
                                                 @((np.transpose(X))@yd))

def output(v,deg,w):
    return (w@(np.transpose(polybasisfun(v,deg))))
    
yp = np.apply_along_axis(output,1,data,deg=degree,w=w)
yvp = np.apply_along_axis(output,1,validata,deg=degree,w=w)
ytp = np.apply_along_axis(output,1,testdata,deg=degree,w=w)
print('ERMS_training = %f'%(np.linalg.norm((yp-yd),2)/np.sqrt(train)))
print('ERMS_test = %f'%(np.linalg.norm((ytp-yt),2)/np.sqrt(test)))
print('ERMS_valid = %f'%(np.linalg.norm((yvp-yv),2)/np.sqrt(valid)))
x = np.linspace(min(data[:,0]),max(data[:,0]),200)
y = np.linspace(min(data[:,1]),max(data[:,1]),200)
X,Y=np.meshgrid(x,y)
Z = (np.array([output([i,j],degree,w) for i,j in  
     zip(np.ravel(X),np.ravel(Y))])).reshape(Y.shape)
fig = plt.figure(figsize=(16,12))
ax = plt.axes(projection='3d')
ax.plot_surface(X,Y,Z,color='red', alpha=0.2,)
ax.scatter(data[:,0],data[:,1],yd,alpha=1,color='black',linewidths=1)
ax.set_xlabel('X-axis(x1)',fontsize = 15) 
ax.set_ylabel('Y-axis(x2)',fontsize = 15) 
ax.set_zlabel('Z-axis(y)',fontsize = 15)
ax.set_title("batch size = {} | degree = {} | lambda = {}".format(batch,degree,
             lamb),fontsize = 20)
plt.show()

fig1 = plt.figure(figsize=(9,9))
plt.scatter(yd,yp,marker='x')
plt.title('training',fontsize=20)
plt.xlabel('target',fontsize=20)
plt.ylabel('predicted',fontsize=20)
fig2 = plt.figure(figsize=(9,9))
plt.scatter(yt,ytp,marker='x')
plt.title('test',fontsize=20)
plt.xlabel('target',fontsize=20)
plt.ylabel('predicted',fontsize=20)
fig3 = plt.figure(figsize=(9,9))
plt.scatter(yv,yvp,marker='x')
plt.title('validation',fontsize=20)
plt.xlabel('target',fontsize=20)
plt.ylabel('predicted',fontsize=20)







\end{minted}
